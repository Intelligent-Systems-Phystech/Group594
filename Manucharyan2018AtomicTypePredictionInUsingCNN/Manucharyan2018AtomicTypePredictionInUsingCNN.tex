\documentclass[12pt,twoside]{article}
\usepackage{jmlda}
\usepackage{hyperref}
\usepackage{bbm}
\usepackage{amssymb}
%\NOREVIEWERNOTES
\title
    [] % Краткое название; не нужно, если полное название влезает в~колонтитул
    {Предсказание свойств и типов атомов в молекулярных графах при помощи сверточных сетей}
\author
    [] % список авторов для колонтитула; не нужен, если основной список влезает в колонтитул
    {Манучарян Вардан} % основной список авторов, выводимый в оглавление
    [Манучарян Вардан] % список авторов, выводимый в заголовок; не нужен, если он не отличается от основного
\thanks
    {Научный руководитель:  Стрижов~В.\,В. 
		 Задачу поставил:  Стрижов~В.\,В.
     Консультанты:  Грудинин~С.,~Кадукова~М.}
\email
    {manucharyan.va@phystec.edu}
\organization
    {Московский физико-технический институт (государственный университет), Москва}
\abstract
    {Статья посвящена определению типов атомов и валентности в молекулярных графах при помощи методов машинного обучения. Предлагается использовать свёрточные нейронные сети (cNN), обученные на 3D структуре молекулярных графов. Для обучения используется <…> датасет, в котором определены тип атома (около 150 классов), гибридизация атома (4 класса) и тип связи (5 классов). Результат работы нового алгоритма будет сравниваться с Knodle \cite{article1}.


\bigskip
\textbf{Ключевые слова}: \emph {машинное обучение, классификация,
свёрточные нейронные сети, молекулярные графы}.}

\begin{document}
\maketitle
%\linenumbers
\section{Введение}
\subsection{}
	В этой работе мы хотим научиться эффективно классифицировать атомы, для того чтобы предсказывать взаимодействие молекул. Это важно во многих вычислительных методах в медицине и биологии, например, виртуальный скрининг при разработке новых лекарств \cite{article2}. При этом молекулы представимы в виде трёхмерных молекулярных графов, что позволяет использовать методы машинного обучения на графах.
	
	“В последние годы были предложены несколько алгоритмов. Самые ранние работы основывались на простых геометрических соображениях, использующих длины связей и валентные углы \cite{article3}. Позже стали учитывать функциональные группы \cite{article4}, гибридизацию и заряд атома \cite{article5,article6,article7,article8}. Чтобы уменьшить влияние ошибок в экспериментальном определении структуры были использованы некоторые подходы: максимальное взвешенное паросочетание \cite{article9,article10}, поиск структуры Льюиса \cite{article11}, максимизация \cite{article12} или минимизация \cite{article13,article14} некой метрики.”
	
	Также используют модели, основанные на свёрточных нейронных сетях (cNN) на графах \cite{article15,article16}.
	
	В данной работе предлагается <..>.
	
	Сравниваться данный алгоритм будет с алгоритмом, реализованным в библиотеке по распознаванию типов атомов Knodle \cite{article1}, основанным на мультиклассовой классификации при помощи метода опорных векторов.
	
\subsection{Обзор функций, используемых для конволюции}
f - произвольная функция, например:
\begin{itemize}
\item $f(z) = \frac{1}{1+e^{-z}}$
\item $f(z) = z I(z > 0)$
\item $f(z) = log(1+exp(x))$
\item для a < 0 (подбирается в ходе обучения сети) 
	\begin{equation*}
	f(z) = 
	 \begin{cases}
		 z &\text{z > 0}\\
		 az &\text{z < 0}
	 \end{cases}
	\end{equation*}
\end{itemize}
g - произвольная коммутативная функция, например:
\begin{itemize}
\item $g(z_1,\dots, z_n) = \sum{z_i}$
\item $g(z_1,\dots, z_n) = \prod{z_i}$
\item $g(z_1,\dots, z_n) = \max{z_i}$
\item $g(z_1,\dots, z_n) = \frac{\sum{z_i}}{n}$
\end{itemize}
\paragraph{Обзор представления молекул}
\begin{itemize}
\item Z-matrix
\item матрица типов связи, матрица минимальных расстояний, матрица состояния в одном кольце вершин
\item 
\end{itemize}

\section{Постановка задачи}
\subsection{Описание выборки}
15000 молекул в формате mol2 из базы данных PDBBindCN. Для каждой молекулы определена матрица смежности G и матрица расстояний D (длины кратчайших путей между атомами). Так же имеются матрица длин связей, углов и двугранных углов. Для каждого атома определены следующие дескрипторы:
\begin{itemize}
\item название элемента
\item электроотрицательность
\item гибридизация
\item тип
\item тип2
\item включение в кольце
\item смешанное произведение векторов связи этой вершины
\end{itemize}

На основе этих данных молекулу можно описать матрицей NxD0 (каждая строчка соответствует атому, D0 - количество признаков) и несколькими матрицами NxN (для кажого парного признака своя матрица).

\subsection{Архитектура сети}
	В сети будет два вида слоёв: атомный и парный. Первый суть 2-мерная матрица, где каждому атому соответствует строка. Второй слой суть 3-мерная матрица, где каждой паре атомов соответствует строка.
	
	\textbf{Определение 2.1} Пусть x - атомный слой, a - атом, тогда $A^x_a$ - значение атома a в слое x. Аналогично, y - парный слой, $(a, b)$ - пара атомов,тогда $P^y_{(a, b)}$ - значение пары $(a, b)$ слое y.
	
	Пусть $f(z) = z I(z > 0)$, $g(z_1,..., z_n) = \sum{z_i}$, $x, x_1,...,x_n$ - слои. Параметры: $c, w_1,\dots,w_n$. Опишем несколько операций, с помощью которых можно получать одни слои из других:
	\begin{itemize}
	\item $(A\rightarrow A)$: $A^y_a = f(с+\sum_{i=1}^n {w_i A^{x_i}_a})$
	\item $(P\rightarrow P)$: $P^y_{(a, b)} = f(с+\sum_{i=1}^n {w_i P^{x_i}_{(a, b)}})$
	\item $(P\rightarrow A)$: $A^y_a = g(f(c + w_1 P^x_{(a, b)}), f(c + w_1 P^x_{(a, c)}), f(c + w_1 P^x_{(a, d)}),...)$, где g вычисляется для всех пар, содержащих a
	\item $(A\rightarrow P)$: $P^y_{(a, b)} = g(f(c + w_1 A^x_a + w_2 A^x_b), f(c + w_1 A^x_b + w_2 A^x_a)))$
	\end{itemize}
	
	\textbf{Лемма 1} Эти операции поддерживают следующий инвариант: если применить ко входу перестановку $\sigma$, то $\forall$ слоёв x, y: $A^x, P^y$ переставляются согласно перестановке $\sigma$
	
	Опишем, как с помощью этих операций получить из k-ого атомного и парного слоёв (k+1)-ые. А именно, пусть есть $A^k$ и $P^k$. Сначала получим промежуточные слои $$A^{k'} = (A\rightarrow A)(A^k), A^{k''}=(P\rightarrow A)(P^k), P^{k'}=(A \rightarrow P)(A^k), P^{k''}=(P\rightarrow P)(P^k),$$ и уже используя их получим (k+1)-ые слои: $$A^{k+1} = (A\rightarrow A)(A^{k'}, A^{k''}), P^{k+1} = (P\rightarrow P)(P^{k'}, P^{k''})$$
Проделав эту процедуру несколько раз, получим финальный атомный слой A.

\subsubsection{Параметры сети}
Параметрами сети являются:
\begin{itemize}
\item параметры функции f: $c, w_1,\dots,w_n$
\item глубина сети
\item глубина конволюции (?непонятно что это)
\item способ получения молекулярных признаков
\item метод оптимизации параметров
\end{itemize}
\subsection{Формальная постановка задачи}
	Пусть  $\mathfrak{G} = \{\mathfrak{s_1},...,\mathfrak{s_m}\}$ - множество атомов в различных молекулах. Пусть $\textbf{y} = \{y_1,...,y_\}$ - типы атомов.

	Пусть $G = \{g_1,...,g_n\}$ - набор функций, таких что $\forall i \forall j$ $g_j$ отображает $\mathfrak{s_i}$ в (i, j) элемент матрицы X: $$g_j: (b_j, \mathfrak{s_i})\rightarrow x_{ij} \in \mathbb{R}^1,$$ где $b_j$ - набор параметров для $g_j$.
	
	Определим $$f(w, X) = \frac{1}{1+exp(-Xw)},$$ где оптимальные параметры $\hat{w}$ минимизируют функцию потерь $$\hat{w}=arg\min\limits_{w} S(w|f, X, y),$$ где $$S(w|f, X, y)=-ln(\sum_{i=1}^m {y_i logf(x_i, w) + (1-y_i)log(1-f(x_i, w))})$$

\begin{thebibliography}{1}
\bibitem{article1}
    \BibAuthor{Maria Kadukova, Sergei Grudinin}
    \BibTitle{Knodle: A Support Vector Machines-Based Automatic Perception of Organic Molecules from 3D Coordinates}~//
    \BibJournal{Journal of Chemical Information and Modeling, American Chemical Society}, 2016, 56 (8), pp.1410-1419.
\bibitem{article2}
    \BibAuthor{Bohdan Waszkowycz, David E Clark, and Emanuela Gancia}
    \BibTitle{Outstanding challenges in protein–ligand docking and structure-based virtual screening}~//
    \BibJournal{Wiley Interdiscip. Rev.: Comput. Mol. Sci.}, 1(2):229-259, 2011.
\bibitem{article3}
    \BibAuthor{Jon C Baber and Edward E Hodgkin}
    \BibTitle{Automatic assignment of chemical connectivity to  organic m the Cambridge structural database}~//
    \BibJournal{J. Chem. Inf. Comput. Sci.}, 32(5):401–406, 1992.
\bibitem{article4}
    \BibAuthor{Manfred Hendlich, Friedrich Rippmann, and Gerhard Barnickel}
    \BibTitle{Bali: Automatic assignment of bond and atom types for protein ligands in the brookhaven protein databank}~//
    \BibJournal{J. Chem. Inf. Comput. Sci.}, 37(4):774–778, 1997.
\bibitem{article5}
    \BibAuthor{Elke Lang, Claus-Wilhelm von der Lieth, and Thomas Forster}
    \BibTitle{Automatic assignment of bond orders based on the analysis of the internal coordinates of molecular structures}~//
    \BibJournal{Anal. Chim. Acta}, 265(2):283–289, 1992.
\bibitem{article6}
    \BibAuthor{Yuan Zhao, Tiejun Cheng, and Renxiao Wang}
    \BibTitle{Automatic perception of organic molecules based on essential structural information}~//
    \BibJournal{J. Chem. Inf. Model.}, 47(4):1379–1385, 2007.
\bibitem{article7}
    \BibAuthor{Daan MF van Aalten, R Bywater, John BC Findlay, Manfred Hendlich, Rob WW Hooft, and Gert Vriend}
    \BibTitle{Prodrg, a program for generating molecular topologies and unique molecular descriptors from coordinates of small molecules}~//
    \BibJournal{J. Comput.-Aided Mol. Des.}, 10(3):255–262, 1996.
\bibitem{article8}
    \BibAuthor{Qian Zhang, Wei Zhang, Youyong Li, Junmei  Wang, Liling Zhang, and Tingjun Hou}
    \BibTitle{A rule based algorithm for automatic bond type perception}~//
    \BibJournal{J. Cheminf.}, 4(1):1–10, 2012. 
\bibitem{article9}
    \BibAuthor{Paul Labute}
    \BibTitle{On the perception of molecules from 3d atomic coordinates}~//
    \BibJournal{J. Chem. Inf.  Model.}, 45(2):215–221, 2005. 
\bibitem{article10}
    \BibAuthor{Gerd Neudertand Gerhard Klebe}
    \BibTitle{Fconv: Format conversion, manipulation and feature computation of molecular data}~//
    \BibJournal{Bioinformatics}, 27(7):1021–1022, 2011.
\bibitem{article11}
    \BibAuthor{Matheus Froeyen and Piet Herdewijn}
    \BibTitle{Correct bond order assignment in a molecular  framework using integer linear programming with application to molecules where only non-hydrogen atom coordinates are available}~//
    \BibJournal{J. Chem. Inf. Model.}, 45(5):1267–1274, 2005.
\bibitem{article12}
    \BibAuthor{Sascha Urbaczek, Adrian Kolodzik, Inken Groth, Stefan Heuser, and Matthias Rarey}
    \BibTitle{Reading pdb: Perception of molecules from 3d atomic coordinates}~//
    \BibJournal{J. Chem. Inf. Model.}, 53(1):76–87, 2012.
\bibitem{article13}
    \BibAuthor{unmei Wang, Wei Wang, Peter A Kollman, and David A Case}
    \BibTitle{Automatic atom type and bond type perception in molecular mechanical calculations}~//
    \BibJournal{J. Mol. Graphics Modell.}, 25(2):247–260, 2006.
\bibitem{article14}
    \BibAuthor{Anna Katharina Dehof, Alexander Rurainski, Quang Bao Anh Bui, Sebastian Bocker, Hans-Peter Lenhof, and Andreas Hildebrandt}
    \BibTitle{Automated bond order assignment as an optimization problem}~//
    \BibJournal{Bioinformatics}, 27(5):619–625, 2011.
\bibitem{article15}
    \BibAuthor{Mathias Niepert, Mohamed Ahmed, Konstantin Kutzkov}
    \BibTitle{Learning Convolutional Neural Networks for Graphs}~//
    \BibJournal{}, 2016
\bibitem{article16}
    \BibAuthor{Steven Kearnes, Kevin McCloskey, Marc Berndl, Vijay Pande, Patrick Riley}
    \BibTitle{Molecular Graph Convolutions: Moving Beyond Fingerprints}~//
    \BibJournal{}, 2016
\bibitem{article17}
    \BibAuthor{Duvenaud DK, Maclaurin D, Iparraguirre J, Bombarell R, Hirzel T, Aspuru-Guzik A, Adams RP}
    \BibTitle{Convolutional networks on graphs for learning molecular fingerprints}~//
    \BibJournal{Advances in neural information processing systems, pp 2224–2232}, 2015
\bibitem{article18}
    \BibAuthor{Lusci A, Pollastri G, Baldi P}
    \BibTitle{Deep architectures and deep learning in chemoinformatics: the prediction of aqueous solubility for drug-like molecules}~//
    \BibJournal{J Chem Inf Model 53(7): 1563–1575}, 2013	
\bibitem{article19}
    \BibAuthor{Merkwirth C, Lengauer T }
    \BibTitle{Automatic generation of complementary descriptors with molecular graph network}~//
    \BibJournal{J Chem Inf Model 45(5):1159–1168}, 2005
\end{thebibliography}

% Решение Программного Комитета:
%\ACCEPTNOTE
%\AMENDNOTE
%\REJECTNOTE
\end{document}
